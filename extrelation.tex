%% extrelation.tex
%% Copyright 2017 F. Pijcke
%
% This work may be distributed and/or modified under the
% conditions of the LaTeX Project Public License, either version 1.3
% of this license or (at your option) any later version.
% The latest version of this license is in
%   http://www.latex-project.org/lppl.txt
% and version 1.3 or later is part of all distributions of LaTeX
% version 2005/12/01 or later.
%
% This work has the LPPL maintenance status `maintained'.
% 
% The Current Maintainer of this work is F. Pijcke.
%
% This work consists of the files extrelation.dtx and extrelation.tex

\documentclass{ltxdoc}

\usepackage{extrelation}
\usepackage{booktabs}
\usepackage{listings}

\newcommand{\example}[1]{
  \indent Code : \texttt{\detokenize{#1}} \\
  \indent Result : #1
}

\title{The \textsf{extrelation} package}
\author{Fabian Pijcke}
\date{2017/03/01 --- v0.1}

\begin{document}

\maketitle

\extrelation{Relation}{A, B; C, D}{a, b, c, d; e, f, g, h}

\begin{abstract}
  This package supports research in the field of relational databases by
  providing simple commands and environments to produce relations given by
  extension. It supports both vertical partitioning (data-wise) and horizontal
  partitioning of the attributes and is easy to use if only basic relations are
  needed.
\end{abstract}

\section{Introduction}

Research papers in the field of relational databases often make use of
relations given by their extension. Especially within examples or when
illustrating proofs working on database construction. Usually producing such
relations is a hassle and inconsistencies in style come easily. This package
addresses both issues.

\section{Simple Relations}

Typesetting a relation composed of $N$ attributes and some $N$-tuples is as
easy as what follows.:

\example{\extrelation{}{A, B, C}{1, 2, 3 ; 4, 5, 6}}

The attributes are to be placed as the second mandatory argument and must be
comma-separated. The tuples form the third and last attribute and are a
semicolon-separated list of tuples, which themselves are colon-separated lists
of values.

\subsection{Relation Name}

The first mandatory attribute allows one to set a relation name. You should not
put any \LaTeX code here, for stylizing is done through the \texttt{relfun}
option (see Section~\ref{sec:funs}

\example{\extrelation{Emp}{Name, Salary}{A, 42; B, 21}}

If the first attribute is empty, then no column is generated for the relation
name. This may be important to know if you use the \texttt{extrelationenv}
environment without the \texttt{extrelationrow} utility command.

\subsection{Attribute Names}

The second mandatory argument is a list of comma-separated attribute names.
Again no \LaTeX code should be put here.

\subsection{Tuples}

The last mandatory argument contains the tuples to be put in the relation. You
should also avoid to put \LaTeX code here, but this can be needed if you want
subscripts for example. Be sure that the style (see Section~\ref{sec:funs}) do
not conflicts with your values in such case.

\section{Horizontal Partitioning of the Attributes} \label{sec:attrpart}

It is not unusual to deal with relations that are horizontally split in two or
more parts. For example, some papers need to distinct the key attributes from
the other attributes. They usually do so by underlining the attributes that are
part of the key. One can typeset a relation in which the attributes $A$ and $B$
are in the key, while attribute $C$ is the only attribute outside the key, as
follows:

\example{\extrelation[attrfuns={\underline{\v}, \v}]{}{A, B; C}{a, b, c}}

Usually, one would then create a macro that supports this type of relation:

\verb!\newcommand{\keyrel}{\extrelation[attrfuns={\underline{\v}, \v}]}!

More informations on the \texttt{attrfuns} option can be found in
Section~\ref{sec:funs}.

\section{Vertical Partitioning}

Some authors find useful to vertically partition the relation. This is done
using the environment instead of the command. The following example uses a
dashed line to split up the tuples of the relation in two parts.

\indent Code: \\
\texttt{\detokenize{\[\begin{extrelationenv}{Relation}{A, B, C}}} \\
\texttt{\detokenize{\extrelationrows{a, b, c; a, d, e} \\}} \\
\texttt{\detokenize{\cmidrule{2-4}}} \\
\texttt{\detokenize{\extrelationrows{f, g, h}}} \\
\texttt{\detokenize{\end{extrelationenv}\]}}

\indent Result:
\[\begin{extrelationenv}{Relation}{A, B, C}
  \extrelationrows{a, b, c; a, d, e} \\
  \cmidrule{2-4}
  \extrelationrows{f, g, h}
\end{extrelationenv}\]

Inside the \texttt{extrelationenv} environment, you are in an \texttt{array}
environment. However we do not recommend using this fact to construct rows by
yourself, as it would potentially break the visual coherence of your relations.

\section{Options Reference}

The following options may be set globally using the \texttt{exrsetdefault}
command, or locally within the \texttt{extrelation} and
\texttt{extrelationrows} command or the \texttt{extrelationenv} environment.

\subsection{toprule, bottomrule}

These options allow to put toprule and bottomrule.

\example{\extrelation[toprule,bottomrule]{}{A, B}{a, b; c, d}}

\subsection{leftrule}

This option allows to put a vertical rule between the relation name and the
relation content.

\example{\extrelation[leftrule]{Rel}{A, B}{a, b; c, d}}

\subsection{pars}

This option allows to parenthesize the relation attributes.

\example{\extrelation[pars]{r = }{A, B}{a, b; c, d}}

\subsection{relfun, attrfuns, confun} \label{sec:funs}

These options allow to change the style of the elements composing the relation.
Each function can make use of the command \texttt{v}. The \texttt{attrfuns}
option is particular in the sense it can be made of several functions, each of
which will be used by one part of the attributes (parts are separated by a
semicolon, see Section~\ref{sec:attrpart}).

The \texttt{relfun} applies to the relation name, defaulting to
\texttt{\detokenize{\textsc{\v}}}.

\example{\extrelation[relfun={\v}]{Rel}{A, B}{a, b; c, d}}

The \texttt{attrfuns} applies to the attribute names, defaulting to
\texttt{\detokenize{\mathbf{\v}}}. Beware that any attribute in a part after
the last function defined in \texttt{attrfuns} will not be typeset.

\example{\extrelation[attrfuns={\underline{\v},\v}]{}{A; B}{}}

The \texttt{confun} applies to the values (con stands for constants),
defaulting to \texttt{\detokenize{\mathsf{\v}}}.

\example{\extrelation[confun=\v]{Rel}{A, B}{a, b; c, d}}

\end{document}
